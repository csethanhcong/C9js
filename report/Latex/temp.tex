\documentclass[12pt]{article}
\usepackage[utf8]{vietnam}
\begin{document}
\tableofcontents{}

\newpage
\section{Giới thiệu đề tài}
\subsection{Bối cảnh}

Trực quan dữ liệu, được hiểu là cách dùng hình ảnh để biểu diễn thông tin, ngày nay đã trở nên phổ dụng bởi những lợi ích nó mang lại cho doanh nghiệp. Nó cung cấp mạnh mẽ những khả năng về hiển thị cũng như cách truyền đạt những thông điệp. 

Mặc cho tiềm năng của nó, Trực quan dữ liệu vẫn chưa được đánh giá cao bởi sự thiếu hiểu biết về nó. Nhiều xu hướng ngày nay về Trực quan dữ liệu thực sự đã gây ra những tác dụng phụ mang tính tiêu cực, đó là sự nhầm lẫn thay vì thấu hiểu thực sự về Trực quan dữ liệu. 

Ngày nay trong lĩnh vực kinh doanh thông minh (Business Intelligence) không có gì có thể mang chúng ta lại gần hơn với những hứa hẹn về sự tương tác thông minh hơn là Trực quan dữ liệu. Nhưng điều này chỉ xảy ra khi chúng ta thực sự hiểu và dùng nó một cách đúng đắn. Và chúng ta phải thực sự hành động và vứt bỏ những quan niệm chưa đúng về Trực quan dữ liệu.

Trực quan dữ liệu ngày càng đóng vai trò quan trọng trong kinh doanh thông minh. Như sử dụng trong nghiên cứu, trong những công việc liên quan tới xử lý dữ liệu, giao dịch bởi người dùng phổ thông, và dùng với tỷ lệ ngày càng tăng trong lực lượng lao động trí óc, đặc biệt là các nhà phân tích. Đó là những tin mang tính khởi sắc. Bên cạnh đó, vẫn còn những mặt hạn chế, trong giới doanh nghiệp, Trực quan dữ liệu vẫn còn bị bỏ ngỏ, hiểu nhầm, sử dụng chưa hiệu quả, và thường bị làm sai lệch đi bởi các nhà cung cấp, sản xuất và bán phần mềm trực quan. 

\subsection{Mục tiêu}



\end{document}